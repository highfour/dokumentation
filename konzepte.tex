\section{Konzepte und Kriterien}

\subsection{Spielkonzept}

Wir haben uns dazu entschieden das Spiel nicht wie klassische Rennspiele mit verschiedenen Strecken, aber den gleichen Fahrzeugen zu jeder Strecke zu gestalten, sondern legen unser Augenmerk auf einzelne Level, die klar voneinander abgetrennt sind.\\
\\
Die Level beschreiben in ihrer Reihenfolge eine Art Evolution durch verschiedene Größenstufen. So beginnt man auf einer sehr kleinen Ebene beim Fahren mit Kleinstkörpern innerhalb der Blutbahn. Im nächsten Level steuert man einen Käfer über eine blättrige Strecke gefolgt von Spielzeugautos, welche im 3. Level über eine Picknickdecke rasen. Menschengroß wird es im nächsten Level, wenn es heißt auf einem Jet Ski über Wasser zu rasen. Zu guter Letzt steuert man Asteroiden durch Planetenumlaufbahnen und weicht den Hindernissen des Weltalls aus.\\
\\
Durch den stark unterschiedlichen Aufbau der Level erhält jedes Level stilistische wie auch spielerische Eigenheiten (Levelspezifische Powerups, Fallen, Hindernisse, abweichende Steuerung), welche das jeweilige Level zu einer neuen Spielerfahrung machen werden so dass das die Neugier des Spielers geweckt sein sollte alle Level auch kennen lernen zu wollen.\\
\\
Die Steuerung erfolgt durch ein Android Gerät und soll möglichst intuitiv ausfallen. Ein wichtiges Augenmerk ist auch die Reaktionszeit zwischen Steuerelement und dem Spiel auf dem Desktop so gering wie möglich zu halten. Mehr dazu im Kapitel 4 Interaktionskonzept.\\

\subsubsection{Level 1 - Blutbahn}

\subsubsection{Level 2 - Käfer}

In unserem zweiten Level geht es raus in die Natur ins Reich der Insekten. Die Spieler übernehmen die Kontrolle über einen Käfer und steuern diesen durch einen Parcour aus Blättern. Leichter gesagt als getan, denn die Blätter sollen an vielen Stellen löchrig oder gar angerissen sein. Wer von der Strecke abweicht und über den Rand fährt oder durch ein Loch fällt, der wird am zuletzt durchfahrenen Checkpoint zurückgesetzt. Über den Streckenverlauf sollen ausreichend Checkpoints verteilt werden um den Spielspaß bei vermehrten Unfällen nicht zu sehr zu trüben. Die Käfer werden die gewählte Farbe des Spielers auf ihrem Rücken enthalten und sich mit Krabbelanimationen fortbewegen. Als Besonderheiten des Levels wären nasse Stellen auf den Blättern möglich, auf denen man entlangrutsch oder besonders raue Stellen, welche den Spieler verlangsamen.

\subsubsection{Level 3 - Picknick}

\subsubsection{Level 4 - Jetski}

Das Rennen in Level 4 wird auf dem Wasser ausgetragen. Die Spieler fahren auf Jetskis über einen Wasserparcours. Die Strecke ist durch Checkpoints abgesteckt, welche sich zwischen Bojen befinden. Der größte Teil des Wassers ist frei befahrbar, allerdings gibt es auch einige Hindernisse, welche umfahren werden müssen (Boote, Schwimmer, Stege, etc.). Ein paar dieser Hindernisse werden sich auch bewegen (Kann-Kriterium).

\subsubsection{Level 5 - Asteroiden}

\subsection{Musskriterien}

\subsection{Kannkriterien}

\begin{itemize}
\item Bewegliche Hindernisse
\item Spielerfoto über Kamera aufnehmen
\item Kollisionen zwischen den Fahrzeugen
\end{itemize}

