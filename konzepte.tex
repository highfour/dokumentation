\section{Konzepte und Kriterien}

\subsection{Spielkonzept}

Wir haben uns dazu entschieden das Spiel nicht wie klassische Rennspiele mit verschiedenen Strecken, aber den gleichen Fahrzeugen zu jeder Strecke zu gestalten, sondern legen unser Augenmerk auf einzelne Level, die klar voneinander abgetrennt sind.\\
\\
Die Level beschreiben in ihrer Reihenfolge eine Art Evolution durch verschiedene Größenstufen. So beginnt man auf einer sehr kleinen Ebene beim Fahren mit Kleinstkörpern innerhalb der Blutbahn. Im nächsten Level steuert man einen Käfer über eine blättrige Strecke gefolgt von Spielzeugautos, welche im 3. Level über eine Picknickdecke rasen. Menschengroß wird es im nächsten Level, wenn es heißt auf einem Jet Ski über Wasser zu rasen. Zu guter Letzt steuert man Asteroiden durch Planetenumlaufbahnen und weicht den Hindernissen des Weltalls aus.\\
\\
Durch den stark unterschiedlichen Aufbau der Level erhält jedes Level stilistische wie auch spielerische Eigenheiten (Levelspezifische Powerups, Fallen, Hindernisse, abweichende Steuerung), welche das jeweilige Level zu einer neuen Spielerfahrung machen werden so dass das die Neugier des Spielers geweckt sein sollte alle Level auch kennen lernen zu wollen.\\
\\
Die Steuerung erfolgt durch ein Android Gerät und soll möglichst intuitiv ausfallen. Ein wichtiges Augenmerk ist auch die Reaktionszeit zwischen Steuerelement und dem Spiel auf dem Desktop so gering wie möglich zu halten. Mehr dazu im Kapitel 4 Interaktionskonzept.\\

\subsubsection{Level 1 - Blutbahn}

Das erste Level fängt ganz klein an. Als Blutkörperchen soll man sich auf Blutbahnen durch das menschliche Innere bewegen. Dabei gilt es bestimmten Hindernissen wie Bakterien und Viren auszuweichen. Der Verlauf der Blutbahnen offenbart sich als wirres Netz mit möglichen Abkürzungen auf dem Weg zum Ziel.

\subsubsection{Level 2 - Käfer}

In unserem zweiten Level geht es raus in die Natur ins Reich der Insekten. Die Spieler übernehmen die Kontrolle über einen Käfer und steuern diesen durch einen Parcour aus Blättern. Leichter gesagt als getan, denn die Blätter sollen an vielen Stellen löchrig oder gar angerissen sein. Wer von der Strecke abweicht und über den Rand fährt oder durch ein Loch fällt, der wird am zuletzt durchfahrenen Checkpoint zurückgesetzt. Über den Streckenverlauf sollen ausreichend Checkpoints verteilt werden um den Spielspaß bei vermehrten Unfällen nicht zu sehr zu trüben. Die Käfer werden die gewählte Farbe des Spielers auf ihrem Rücken enthalten und sich mit Krabbelanimationen fortbewegen. Als Besonderheiten des Levels wären nasse Stellen auf den Blättern möglich, auf denen man entlangrutsch oder besonders raue Stellen, welche den Spieler verlangsamen.

\subsubsection{Level 3 - Picknick}

Das einzige Level im Spiel, in dem man tatsächlich ein Auto steuert. Mit Spielzeugautos fährt man über eine Picknickdecke. Es gilt Teller, Besteck und umgefallenen Tassen zu umfahren. Als weitere Hindernisse sind Saucenflecken denkbar, auf denen man ausrutscht und kurz die Kontrolle verliert. Außerhalb der Picknickdecke erschweren Offroad Parts den Fahrern das Leben.

\subsubsection{Level 4 - Jetski}

Das Rennen in Level 4 wird auf dem Wasser ausgetragen. Die Spieler fahren auf Jetskis über einen Wasserparcours. Die Strecke ist durch Checkpoints abgesteckt, welche sich zwischen Bojen befinden. Der größte Teil des Wassers ist frei befahrbar, allerdings gibt es auch einige Hindernisse, welche umfahren werden müssen (Boote, Schwimmer, Stege, etc.). Ein paar dieser Hindernisse werden sich auch bewegen (Kann-Kriterium).

\subsubsection{Level 5 - Asteroiden}

Im letzten Level des Spiels steuern die Spieler Raumschiffe um Planeten herum durch das Weltall. Die Gravitationsfelder der Planeten stellen dabei die Rennstrecke dar, haben jedoch keinen Einfluss auf das Fahrverhalten der Fahrzeuge. Die Herausforderung an diesem Level stellen die Asteroiden, die als Hindernisse in verschiedenen Geschwindigkeiten die Rennstrecke entlang kreisen, und die sich ständig um die Planeten herumgabelnde Strecke dar.

\subsection{Musskriterien}

\begin{itemize}
\item Menübildschirm zur Auswahl der Strecke und Verbindung der Spieler auf der Desktop-Seite
\item Verbindung der Mobilgeräte zum Desktop-Programm über ein lokales Netzwerk
\item Steuerung der Fahrzeuge über die im Interaktionskonzept beschriebenen Eingabetechniken
\item Kollisionsabfrage an Streckenrändern (nur Rennstrecke befahrbar)
\item Aufsammeln und Anzeige der Items auf dem Mobilgerät
\item Nutzung der Items über die im Interaktionskonzept beschriebenen Eingabetechniken
\item Anzeige von Zeit, Rundenanzahl, aktueller Position
\item Ende eines Rennens mit Anzeige von Zeiten und Platzierungen
\item Mindestens 3 Level vollständig umgesetzt
\end{itemize}

\subsection{Kannkriterien}

\begin{itemize}
\item Bewegliche Hindernisse auf den Strecken
\item Spielerfoto über Kamera aufnehmen
\item Kollisionen zwischen den Fahrzeugen
\item alle 5 Level vollständig umgesetzt
\item Fortschrittsanzeige der einzelnen Renn-Teilnehmer auf der aktuellen Strecke
\item Animierte Fahrzeuge und Streckeninhalte
\item Cup-Modus, in dem verschiedene Strecken hintereinander gefahren werden
\end{itemize}

\newpage
