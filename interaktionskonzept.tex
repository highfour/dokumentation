\section{Interaktions- und Bedienkonzept}

Das Interaktionskonzept unseres Spiels bedient sich hauptsächlich des Beschleunigungssensors bzw. des Gyroskop der genutzten mobilen Geräte. Weiterhin soll eine Eingabe über den Touchscreen erfolgen.\\
Die Steuerung im Spielmenü wird in großen Teilen über den Computer geregelt. Das betrifft z.B. die Auswahl der Strecke und die Nutzung von globalen Buttons. Zur Eingabe wird die Maus oder Tastatur genutzt.\\
Sobald ein Spieler sich mit seinem Mobilgerät zum Computer verbunden hat, wird über das Display seine zugewiesene Farbe dargestellt und es erfolgt eine Namenseingabe über die Bildschrimtastatur. Außerdem kann ein Foto mit der Kamera aufgenommen werden, welches dann als Spielerbild fungiert (Kann-Kriterium). Vor Spielbeginn muss jeder verbundene Spieler auf seinem Touchscreen den \glqq Bereit\grqq\ Button anklicken.\\
Im Spiel soll der Spieler nur selten auf das Mobilgerät schauen und hat seinen Fokus auf der Rennstrecke (welche dann per Beamer oder auf einem großen Bildschirm zu sehen ist). Daher erfolgt die Steuerung des jeweiligen Fahrzeugs über die Neigung der Smartphones oder Tablets. Dabei ist die Lenkrichtung immer direkt mit dem Fahrzeug verbunden. Eine Neigung nach links veranlasst das Fahrzeug nach links zu lenken, auch wenn es auf der Rennstrecke gerade nach unten fährt und somit scheinbar nach rechts lenkt. Für einen Beobachter sieht das Verhalten vielleicht nicht korrekt aus, für den Spieler ist es allerdings intuitiv.\\
Die verschiedenen Items, welche der Spieler aufsammeln und nutzen kann, werden über den Touchscreen aktiviert. Manche Items benötigen nur einen Klick, wobei andere vom Fahrzeug aus nach Vorne und nach Hinten \glqq geschossen\grqq\ werden können. Dazu nutzen wir eine einfache Wischgeste, welche die Richtung bestimmt.
