\section{Grundlegende Spielidee und Ziel}

Unser Team entschied sich zur Entwicklung eines Rennspiels in \textit{top-down} 2D Grafik.
Die Orientierung erfolgt grob an bekannten Spielen wie \textit{Micro Machines} oder \textit{Black Mamba Racing}.

Eine Gruppe von Spielern wird die Möglichkeit haben, anhand der Controller-Applikation auf ihren Androidgeräten auf einer auf dem Beamer/Desktoprechner dargestellten Karte mehrere Runden einer Rennstrecke zu absolvieren. 
Währendessen haben sie die Möglichkeit durch verschiedene aufsammelbare Powerups entweder sich selbst einen Vorteil zu verschaffen oder anderen Spielern Schaden hinzuzufügen.

Die verschiedenen Rennstrecken sind in fünf verschiedene Level gekapselt, die im folgenden noch weiter vorgestellt werden.
Diese können entweder direkt einzeln ausgewählt und gespielt werden oder in einem Cup der Reihenfolge nach befahren werden, wobei der finale Sieger nicht durch ein einzelnes Rennen, sondern in diesem Fall anhand von durch Einzelsiege erhaltene Punkte am Ende des Cups ermittelt wird.
